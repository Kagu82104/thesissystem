\begin{acknowledgementsCH}

所有對於研究提供協助之人或機構,作者都可在誌謝中表達感謝之意。標題使用20pt粗標楷體,並於上、下方各空一行(1.5倍行高,字型12pt空行)後鍵入內容。致謝頁須編頁碼(小寫羅馬數字表示頁碼)。\\\\



\begin{enumerate}[leftmargin=0pt, topsep=0pt, itemsep=0pt, label=\Roman{*}.]

\item 此範本參考下列網站的資料:
\begin{enumerate}[topsep=0pt, itemsep=0pt, label=$\bullet$]
    \item \href{https://code.google.com/p/ntu-thesis-latex-template/}{台大碩博士論文LaTeX範本}
    \item \href{http://exciton.eo.yzu.edu.tw/~lab/latex/latex_note.html}{陳念波老師的元智大學論文樣板}
    \item \href{https://code.google.com/p/ntust-thesis/}{台灣科技大學同學編寫的碩博士論文Latex模板}
\end{enumerate}

\item 原作者參考並修改自下列網站的資料:
\begin{enumerate}[topsep=0pt, itemsep=0pt, label=$\bullet$]
    \item \href{http://www.csie.ntu.edu.tw/~tzhuan/www/resources/ntu/}{如何用 LaTeX 排版臺灣大學碩士論文}\\
    \textemdash 台灣大學論文\LaTeX\ 樣版原創者\href{http://www.csie.ntu.edu.tw/~tzhuan/www/}{黃子桓}的教學網頁
    \item \href{http://www.hitripod.com/blog/2012/05/latex-thesis-template-quick-reference/}{LaTeX 常用語法及論文範本}\\
    \textemdash \href{http://www.hitripod.com/blog/}{Hitripod}所修改的範本,這裡參考了許多他所寫的格式和內容
    \item \href{http://www.cc.ntu.edu.tw/chinese/epaper/0014/20100920_1404.htm}{使用LaTeX做出精美的論文}
    \item \href{http://www.hitripod.com/blog/2011/04/xetex-chinese-font-cjk-latex/}{XeTeX:解決 LaTeX 惱人的中文字型問題}
    \item \href{http://code.google.com/p/ntuthesis/}{台灣大學碩士、博士論文的Latex模板}\\   
\end{enumerate}

\end{enumerate}


%----------- Have a fractal fern? -----------
%\begin{pspicture}
%\psFern[scale=30,maxIter=100000,linecolor=Green]
%\end{pspicture}

\end{acknowledgementsCH} 