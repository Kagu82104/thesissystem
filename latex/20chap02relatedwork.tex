\chapter{相關文獻討論}
\label{c:2}
%==========================================================================================
視覺化演算法在CNN上已經有許多相關的研究,一開始的研究著重於將圖片的特徵,但隨著CNN的快速發展,視覺化已經擴展到解釋CNN的整體架構與運作方式。主要是解析每個演算法的網路架構和演算法的邏輯,其中有幾個較具代表性的方法:\\
Erhan 等人提出 Activation Maximization 來對傳統的淺層網路進行解釋。\\
後來,Simonyan 等人通過將單個 CNN 神經元的最大啟用視覺化合成一個輸入影像模式( input image pattern ),進一步改進了這種方法。\\
後續出現了很多工作都是基於這種方法,再利用不同的正則項進行擴充套件,以提高合成影像模式的可解釋性。\\
Mahendran 等人提出了 Network Inversion 重建基於多個神經元啟用的輸入影像,以此說明每個 CNN 層學習到的綜合特徵圖,揭示了 CNN 網路在網路層層面的內部特徵。\\
Network Inversion 根據特定層的特徵圖中的原始影像重建輸入影像,這可以揭示該圖層所儲存的影像資訊。\\
沒有選擇對輸入影像進行重建以實現特徵視覺化,Zeiler 等人提出了基於反摺積神經網路的視覺化方法(Deconvolutional Neural Network based Visualization, DeconvNet),該方法利用 DeconvNet 框架將特徵圖直接對映到影像維度,利用反摺積 CNN 結構(由反摺積層和反摺積層組成)在特定神經元啟用的原始輸入影像中查詢影像模式。
\\通過直接對映, DeconvNet 可以突出顯示輸入影像中的哪些模式啟用特定神經元,從而直接連結神經元和輸入資料的含義。\\
周博磊等人提出了 Network Dissection based Visualization,它從語義層面對 CNN 進行了解釋。\\
通過引用異構影像資料集——Borden,Network Dissection 可以有效地將輸入影像分割為多個具有各種語義定義的部分,可以匹配六種語義概念(例如場景,目標,部件,材質,紋理和顏色)。\\
由於語義直接代表了特徵的含義,神經元的可解釋性可以顯著提高。\\
以上都是圍繞在以CNN為基礎的視覺化,而鮮少有對其他較進階的CNN演算法進行視覺化分析。
%==========================================================================================

%\section{Manga Vectorization and Manipulation with Procedural Simple Screentone.}
%資料來源:\cite{7399427}\par
%影片:\href{video.mp4}{影片}

%==========================================================================================

%\section{YOLO9000:Better, Faster, Stronger.}
%資料來源:\cite{RedmonF17}\par
%影片:\href{YOLO 9000 Better Faster Stronger.mp4}{影片}

%==========================================================================================
%\section{Deep Residual Learning for Image Recognition. 卷積影像深度學習}
%資料來源:{Microsoft Research.2016 IEEE\cite{DBLP:journals/corr/HeZRS15}}
%影片:\href{Research Talk (in Hindi) Deep Residual Learning for Image Recognition.mp4}{影片}

%==========================================================================================

%\section{Only Look Once, Mining Distinctive Landmarks from ConvNet for Visual Place Recognition.只看一次,在ConvNet找到特殊的地標,用於地點識別}
%資料來源:{2017 IEEE/RSJ International Conference on Intelligent Robots and Systems (IROS)\cite{Chen:etal:IROS2017}}\\
%影片:\href{1331_VI.mp4}{影片}

%==========================================================================================

%\section{Visualizing and Understanding Convolutional Networks}
%資料來源:{Dept. of Computer Science,New York University, USA\cite{DBLP:journals/corr/ZeilerF13}}\\
%影片:\href{1331_VI.mp4}{影片}

%==========================================================================================

%\section{Visualizing Convolutional Neural Networks for Image Classification}
%資料來源:{Dept. of Computer Science,New York University, USA\cite{DBLP:journals/corr/abs-1804-11191}}\\
%影片:\href{1331_VI.mp4}{影片}

%==========================================================================================

%\section{Visualization of Neural Network Predictions for Weather Forecasting}
%資料來源:{COMPUTER GRAPHICS forumVolume 00 (2018), number 0 pp. 1–12 \cite{Roesch2017VisualizationON}}\\
%影片:\href{Visualization of Neural Network Predictions for Weather Forecasting (VMV 2017).mp4}{影片}

%==========================================================================================

%\section{Deep learning for computational biology}
%資料來源:{COMPUTER GRAPHICS forumVolume 00 (2018), number 0 pp. 1–12 \cite{Roesch2017VisualizationON}}\\
%影片:\href{Visualization of Neural Network Predictions for Weather Forecasting (VMV 2017).mp4}{影片}

CNN、R-CNN、Fast R-CNN、Faster R-CNN、Mask R-CNN、SSD、YOLO、YOLOv2、YOLOv3……等,都是屬於使用CNN模型,只要輸入一張圖片,並得到該圖片分類的結果,但每個類型所使用的架構並不相同。

